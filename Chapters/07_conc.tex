\section{Conclusions}
In this work a real-time path planning algorithm was developed. The algorithm is based on a modified RRT and a trajectory-optimization algorithm. The modified RRT allows to compute curvature constrained trajectories. The trajectory-optimization algorithm is based on a simplified dynamics model for the UAV. This dynamics is used to formulate a trajectory-optimization problem that can be solved in real time. This optimization problem is designed in such a way that it aims to minimize an estimate on the mission cost. This estimate is a combination between trajectory-time and energy/fuel consumption. Aiming for reducing the number of design variables, collision-free constraints are assured in intermediate points between way-points, to minimize the number of way-points necessary to describe the trajectory. Results show that this technique improves significantly the computational efficiency of the algorithm.
\par
Both the RRT and the trajectory optimization algorithms were combined into a real-time trajectory-planner. An additional feature was which quickly stops the multi-rotor when a close collision time is predicted. Several simulations using a simplified environment are performed that validate the capability of the algorithm to generate collision free trajectories in partially unknown environments. It was also proven the capability of avoiding moving obstacles, which trajectories are predicted assuming a constant obstacle speed.
\par
The algorithms were also used to plan the trajectories of two agents using a distributed approach. The agents were capable of coordinating among them collision free trajectories.
\par
Finally, it was proven a vehicle with realistic dynamics is capable of following aggressive trajectories generated by the proposed planner. The validation was achieved using the Gazebo simulator and an existing controller.
\par
For future work, the authors propose that the algorithms are expanded for dealing with general convex obstacles. It would also be interesting to expand the algorithms for fixed wing aircraft by, for example, limiting the stall speed and the climb rate.